\paragraph{} En esta secci\'on realizamos una aplicaci\'on que muestra el contenido del i-nodo de
un archivo cuyo $\$HOME/nombre$ es pasado como par\'ametro.

\subsection{Algorítmo}

\paragraph{} El algoritmo verifica primero que la cantidad de parametros recibida sea correcta, si no 
es asi mostrar\'a el siguiente mensaje: 

\paragraph{}\emph{''Error: el formato de entrada es ./datosINodo \$HOME/nombreArchivo''}.\\ 

Si la cantidad de par\'ametros es correcta esta ser\'a la ruta completa del archivo, 
una vez que la tenemos debemos verificar que este exista, si existe muestra la informaci\'on 
del i-nodo y sino nos muestra el siguiente mensaje:

\paragraph{}\emph{''Error:El archivo pasado como par\'ametro no existe''.}\\

La informaci\'on que vamos a mostrar del i-nodo de un archivo es la siguiente:

\begin{itemize}
	\item Ruta completa del archivo
	\item Numero de dispositivo
	\item Numero de i-nodo
	\item Cantidad de links
	\item ID del propietario
	\item ID del grupo del propietario
	\item Tamanio del archivo
	\item Tipo de archivo	
	\item Permisos
		\begin{itemize}
	      		\item Para el propietario
	      		\item Para el grupo
	      		\item Para otros
		\end{itemize}
	\item Ultima fecha modificaci\'on
	\item Fecha de cambio del status
	\item Ultima fecha de acceso
\end{itemize}

\paragraph{} Para obtener la informaci\'on del i-nodo utilizamos la estructura 
$stat$ y la funci\'on \emph{int$ $stat (const char *\_path, struct stat *\_buf)} a la cual 
le pasamos la ruta completa del archivo y un $struc stat$, nos cargar\'a en la 
estructura todo la informaci\'on del i-nodo del archivo. \\
La estructura stat es la que nos provee minix para obtener informaci\'on de un i-nodo y se encuentra en el 
directorio $/usr/include/sys$. \\

A continuaci\'on mostramos la definici\'on de la estructura y las funciones que provee:

\begin{scriptsize} 
\begin{verbatimtab} 

struct stat {
  dev_t st_dev;			/* major/minor device number */
  ino_t st_ino;			/* i-node number */
  mode_t st_mode;		/* file mode, protection bits, etc. */
  short int st_nlink;		/* # links; TEMPORARY HACK: should be nlink_t*/
  uid_t st_uid;			/* uid of the file's owner */
  short int st_gid;		/* gid; TEMPORARY HACK: should be gid_t */
  dev_t st_rdev;
  off_t st_size;		/* file size */
  time_t st_atime;		/* time of last access */
  time_t st_mtime;		/* time of last data modification */
  time_t st_ctime;		/* time of last file status change */
};

/* Function Prototypes. */

_PROTOTYPE( int chmod, (const char *_path, Mode_t _mode)		);
_PROTOTYPE( int fstat, (int _fildes, struct stat *_buf)			);
_PROTOTYPE( int mkdir, (const char *_path, Mode_t _mode)		);
_PROTOTYPE( int mkfifo, (const char *_path, Mode_t _mode)		);
_PROTOTYPE( int stat, (const char *_path, struct stat *_buf)		);
_PROTOTYPE( mode_t umask, (Mode_t _cmask)				);

\end{verbatimtab} 
\end{scriptsize} 
  
\paragraph{} Para verificar que el archivo exista, primero pensamos en abrirlo pero el problema 
es que siempre estariamos modificando la ultima fecha de acceso, entonces se nos ocurri\'o 
crear una estructura stat en la cual seteamos el n\'umero de inodo en -1, despues se llama 
a la funci\'on \\ \emph{ int stat (const char *\_path, struct stat *\_buf)}.\\
De existir el archivo nos modificar\'a la estructura stat y nos cargara la informaci\'on del i-nodo, 
si una vez llamado a este m\'etodo el n\'umero de i-nodo sigue siendo -1 entonces el archivo no existe y mostraremos 
el siguiente mensaje de error: \emph{''Error: El archivo pasado como parametro no existe''.}\\
Si el archivo existe entonces mostramos por consola toda la informaci\'on del i-nodo. 
\paragraph{} Nos encontramos con un problema a la hora de imprimir la fecha ya que c no 
nos provee un print para fechas. Para resolverlo creamos el metodo formatoFecha que nos 
permite imprimir con el formato $YYYY-MM-DD$ $HH:MM:SS$. Este m\'etodo se encuentra en el archivo $datosINodo.c$.
\paragraph{} A continuaci\'on la implemetaci\'on del m\'etodo $datosINodo.c$

\begin{scriptsize} 
\begin{verbatimtab} 

int main( int argc, char** argv)
{

	char cmd[4096]; 	
	
	
	if( (argc <= 1) || (argc >2))
	{
	     	printf("%s\n", "Error: el formato de entrada es ./datosINodo $HOME/nombreArchivo");	
	
	}
	else
	{
		struct stat estr;

		estr.st_ino=65535;
		stat(argv[1],&estr);
		
		if(estr.st_ino==65535) /*seteamos el ino en -1*/	
		{
			printf("%s\n","Error: El archivo pasado como paramtero no existe");
		}	
		
		else {
			printf("Ruta completa del archivo:%s\n",argv[1]);
			printf("Numero de dispositivo: %u\n",estr.st_dev);
			printf("Numero de i-nodo: %u\n",estr.st_ino);
			printf("Cantidad de links: %u\n",estr.st_nlink);
			printf("ID del propietario: %u\n",estr.st_uid);
			printf("ID del grupo del propietario: %u\n",estr.st_gid);
			printf("Tamanio del archivo: %u\n",estr.st_size);


			printf("Tipo de archivo:");	
			if(S_ISDIR(estr.st_mode)) {
				printf("Directorio");
			}
			if(S_ISCHR(estr.st_mode)){ 
				printf("especialdecaracteres");
			}
			if(S_ISBLK(estr.st_mode)){
				printf("especialdebloques");
			}	
			if(S_ISFIFO(estr.st_mode)){
				printf("archivofifo");
			}
			if(S_ISREG(estr.st_mode)){
				printf("regular");
			}

	
			printf("\nPermisos\n");
			printf("\t Para el propietario:");
			if( estr.st_mode & S_IRUSR ){
				printf("r");
			}
			if( estr.st_mode & S_IWUSR ){
				printf("w");
			}
			if( estr.st_mode & S_IXUSR ){
				printf("x");
			}

	
			printf("\n");
			printf("\t Para el grupo:");

			if( estr.st_mode & S_IRGRP ){
				printf("r");
			}
			if( estr.st_mode & S_IWGRP ){
				printf("w");
			}
			if( estr.st_mode & S_IXGRP ){
				printf("x");
			}


			printf("\n");
			printf("\t Para otros:");
			if( estr.st_mode & S_IROTH ){
				printf("r");
			}
			if( estr.st_mode & S_IWOTH ){
				printf("w");
			}
			if( estr.st_mode & S_IXOTH ){
				printf("x");
			}

	
			printf("\n");
			printf("Ultima fecha modificacion:");
			formatoFecha(&(estr.st_mtime));
	
			printf("Fecha de cambios del status:");
			formatoFecha(&(estr.st_ctime));
	
			printf("Ultima fecha de acceso:");
			formatoFecha(&(estr.st_atime));
		}
		
	}
	return 0;
}

\end{verbatimtab} 
\end{scriptsize} 

\subsection{Localizaci\'on del c\'odigo y ejemplos}
En el directorio $/usr/tp/15$ se encuentra el archivo $datosINodo.c$ que contiene el c\'odigo 
antes mencionado. Se encuentra tambi\'en en el mismo directorio el script de compilaci\'on $compex$ 
que genera el ejecutable $datosINodo$. Para usarlo basta con escribir:  
\paragraph{}\emph{./datosINodo \$HOME/nombre}\\

A continuaci\'on mostramos las diferentes pruebas que se realizaron para comprobar el correcto
funcionamiento del programa.

\subsubsection{Ejemplo 1}

\begin{enumerate}
	\item Creamos un archivo $prueba1$ en el directorio $/usr/tp/15$. Luego ejecutamos 
    \emph{./datosInodo /usr/tp/15/prueba1} y obtenemos la siguiente informaci\'on:

\begin{scriptsize} 
\begin{verbatimtab} 
Ruta completa del archivo:/usr/tp/15/prueba1
Numero de dispositivo: 770
Numero de i-nodo: 3409
Cantidad de links: 1
ID del propietario: 0
ID del grupo del propietario: 0
Tamanio del archivo: 43
Tipo de archivo:regular
Permisos
	 Para el propietario:rw
	 Para el grupo:r
	 Para otros:r
Ultima fecha modificacion:Sat 2009-10-03 19:09:58 MET
Fecha de cambios del status:Sat 2009-10-03 19:09:58 MET
Ultima fecha de acceso:Sat 2009-10-03 19:09:58 MET
\end{verbatimtab} 
\end{scriptsize} 


\item Luego hacemos un \emph{chmod 777 /usr/tp/15/prueba1}, ejecutamos nuevamente\\
\emph{./datosInodo /usr/tp/15/prueba1} y obtenemos la siguiente informaci\'on:

\begin{scriptsize} 
\begin{verbatimtab} 
Ruta completa del archivo:/usr/tp/15/prueba1
Numero de dispositivo: 770
Numero de i-nodo: 3409
Cantidad de links: 1
ID del propietario: 0
ID del grupo del propietario: 0
Tamanio del archivo: 43
Tipo de archivo:regular
Permisos
	 Para el propietario:rwx
	 Para el grupo:rwx
	 Para otros:rwx
Ultima fecha modificacion:Sat 2009-10-03 19:09:58 MET
Fecha de cambios del status:Sat 2009-10-03 19:15:07 MET
Ultima fecha de acceso:Sat 2009-10-03 19:09:58 MET
\end{verbatimtab} 
\end{scriptsize} 


\paragraph{} Se puede observar que se modific\'o adem\'as de los permisos, la fecha de cambio de status del archivo.

\item Ahora accedemos al archivo pero no lo modificamos y obtenemos la siguiente informaci\'on:
\begin{scriptsize} 
\begin{verbatimtab} 
Ruta completa del archivo:/usr/tp/15/prueba1
Numero de dispositivo: 770
Numero de i-nodo: 3409
Cantidad de links: 1
ID del propietario: 0
ID del grupo del propietario: 0
Tamanio del archivo: 43
Tipo de archivo:regular
Permisos
	 Para el propietario:rwx
	 Para el grupo:rwx
	 Para otros:rwx
Ultima fecha modificacion:Sat 2009-10-03 19:09:58 MET
Fecha de cambios del status:Sat 2009-10-03 19:15:07 MET
Ultima fecha de acceso:Sat 2009-10-03 19:18:33 MET
\end{verbatimtab} 
\end{scriptsize} 

\item Ahora accemos al archivo lo modificamos y guardamos los cambios y obtenemos la siguiente informaci\'on:
\begin{scriptsize} 
\begin{verbatimtab} 
Ruta completa del archivo:/usr/tp/15/prueba1
Numero de dispositivo: 770
Numero de i-nodo: 3409
Cantidad de links: 1
ID del propietario: 0
ID del grupo del propietario: 0
Tamanio del archivo: 43
Tipo de archivo:regular
Permisos
	 Para el propietario:rwx
	 Para el grupo:rwx
	 Para otros:rwx
Ultima fecha modificacion:Sat 2009-10-03 19:20:47 MET
Fecha de cambios del status:Sat 2009-10-03 19:20:47 MET
Ultima fecha de acceso:Sat 2009-10-03 19:20:47 MET
\end{verbatimtab} 
\end{scriptsize} 

\end{enumerate}

\subsubsection{Ejemplo 2}

\paragraph{} Ahora realizamos una prueba con el archivo $stat.h$ que no creamos nosotros, ejecutamos $./datosINodo$ $/usr/include/sys/stat.h$ y obtenemos la siguiente informaci\'on:

\begin{scriptsize} 
\begin{verbatimtab} 
Ruta completa del archivo:/usr/include/sys/stat.h

Numero de dispositivo: 770
Numero de i-nodo: 450
Cantidad de links: 1
ID del propietario: 2
ID del grupo del propietario: 0
Tamanio del archivo: 3012
Tipo de archivo:regular
Permisos
	 Para el propietario:rw
	 Para el grupo:r
	 Para otros:r
Ultima fecha modificacion:Tue 1996-10-01 13:00:00 MET
Fecha de cambios del status:Sun 2005-03-13 23:15:11 MET
Ultima fecha de acceso:Sat 2009-10-03 18:51:51 MET
\end{verbatimtab} 
\end{scriptsize} 
